\let\negmedspace\undefined
\let\negthickspace\undefined
\documentclass[journal,12pt,twocolumn]{IEEEtran}
\usepackage{cite}
\usepackage{amsmath,amssymb,amsfonts,amsthm}
\usepackage{algorithmic}
\usepackage{graphicx}
\usepackage{textcomp}
\usepackage{xcolor}
\usepackage{txfonts}
\usepackage{listings}
\usepackage{enumitem}
\usepackage{mathtools}
\usepackage{gensymb}
\usepackage[breaklinks=true]{hyperref}
\usepackage{tkz-euclide}
\usepackage{listings}
\begin{document}
\title{AI1110 - Assignment1}
\author{Rayi Giri Varshini - EE22BTECH11215}	
\maketitle
\textbf{Question: 10.13.2.12}
Sushma tosses a coin 3 times and gets tail each time. Do you think that the outcome of next toss will be a tail? Give reasons.
\textbf{Solution:}
As the coin is tossed 3 times and gets a tail each time but it is not necessary that 4th time will be a tail. It may be either tail or head in any
further toss. 
Let X be the random variable for the occarance of tail.
In this binomial distribution, $n = 4$.
 $P(X = x) = \binom{n}{x} q^{n-x} p^x$, where x can be a number from 0 to n, $p = q = \frac{1}{2}.$                                          
 $P(X = 0) = \binom{4}{0} q^{4-0} p^0$ = $\binom{4}{0} (\frac{1}{2})^{4-0} (\frac{1}{2})^0 = (\frac{1}{16}).$
 $P(X = 1) = \binom{4}{1} q^{4-1} p^1$ = $\binom{4}{1} (\frac{1}{2})^{4-1} (\frac{1}{2})^1 = (\frac{1}{4}).$
 $P(X = 2) = \binom{4}{2} q^{4-2} p^2$ = $\binom{4}{2} (\frac{1}{2})^{4-2} (\frac{1}{2})^2= (\frac{3}{8}).$
 $P(X = 3) = \binom{4}{3} q^{4-3} p^3$ = $\binom{4}{3} (\frac{1}{2})^{4-3} (\frac{1}{2})^3 = (\frac{1}{4}).$
 $P(X = 4) = \binom{4}{4} q^{4-4} p^4$ = $\binom{4}{4} (\frac{1}{2})^{4-4} (\frac{1}{2})^4 = (\frac{1}{16}).$
As the coin in unbiased,
Probability of $Head = Tail = \frac{1}{2} $ in every single case.
Hence, the given statement is false.

\end{document}
